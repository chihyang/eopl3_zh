% for code diff
\usepackage[usetwoside=false]{mdframed}
\mdfdefinestyle{codediff}{usetwoside=false,
  skipabove=4\lineskip,
  topline=false,
  rightline=false,
  leftline=true,
  bottomline=false,
  innerleftmargin=0pt,
  innerrightmargin=0pt,
  innertopmargin=0pt,
  innerbottommargin=0pt,
  linewidth=0.8pt,
  leftmargin=-2ex,
  innerleftmargin=2em}
\usepackage{comment}

% This style file is used both for `book' and `report' modes, copied from scribble/book

\renewcommand{\ChapRef}[2]{chapter~#1}
\renewcommand{\ChapRefUC}[2]{Chapter~#1}

% sections
\renewcommand{\Spart}[2]{\part[#1]{#2}}
\renewcommand{\Ssection}[2]{\chapter[#1]{#2}}
\renewcommand{\Ssubsection}[2]{\section[#1]{#2}}
\renewcommand{\Ssubsubsection}[2]{\subsection[#1]{#2}}
\renewcommand{\Ssubsubsubsection}[2]{\subsubsection[#1]{#2}}
\renewcommand{\Ssubsubsubsubsection}[2]{{\bf #2}}

% "star" means unnumbered and not in ToC:
\renewcommand{\Spartstar}[1]{\part*{#1}}
\renewcommand{\Ssectionstar}[1]{\chapter*{#1}}
\renewcommand{\Ssubsectionstar}[1]{\section*{#1}}
\renewcommand{\Ssubsubsectionstar}[1]{\subsection*{#1}}
\renewcommand{\Ssubsubsubsectionstar}[1]{\subsubsection*{#1}}
\renewcommand{\Ssubsubsubsubsectionstar}[1]{\Ssubsubsubsubsection{#1}}

% "starx" means unnumbered but in ToC:
\renewcommand{\Spartstarx}[2]{\Spartstar{#2}\addcontentsline{toc}{part}{#1}}
\renewcommand{\Ssectionstarx}[2]{\Ssectionstar{#2}\addcontentsline{toc}{chapter}{#1}}
\renewcommand{\Ssubsectionstarx}[2]{\Ssubsectionstar{#2}\addcontentsline{toc}{section}{#1}}
\renewcommand{\Ssubsubsectionstarx}[2]{\Ssubsubsectionstar{#2}\addcontentsline{toc}{subsection}{#1}}
\renewcommand{\Ssubsubsubsectionstarx}[2]{\Ssubsubsubsectionstar{#2}\addcontentsline{toc}{subsubsection}{#1}}
\renewcommand{\Ssubsubsubsubsectionstarx}[2]{\Ssubsubsubsubsectionstar{#2}}

% "grouper" is for the 'grouper style variant:
\renewcommand{\Ssubsectiongrouper}[2]{\setcounter{GrouperTemp}{\value{section}}\Ssubsectionstarx{#1}{#2}\setcounter{section}{\value{GrouperTemp}}}
\renewcommand{\Ssubsubsectiongrouper}[2]{\setcounter{GrouperTemp}{\value{subsection}}\Ssubsubsectionstarx{#1}{#2}\setcounter{subsection}{\value{GrouperTemp}}}
\renewcommand{\Ssubsubsubsectiongrouper}[2]{\setcounter{GrouperTemp}{\value{subsubsection}}\Ssubsubsectionstarx{#1}{#2}\setcounter{subsubsection}{\value{GrouperTemp}}}
\renewcommand{\Ssubsubsubsubsectiongrouper}[2]{\Ssubsubsubsubsectionstarx{#1}{#2}}

\renewcommand{\Ssubsectiongrouperstar}[1]{\setcounter{GrouperTemp}{\value{section}}\Ssubsectionstar{#1}\setcounter{section}{\value{GrouperTemp}}}
\renewcommand{\Ssubsubsectiongrouperstar}[1]{\setcounter{GrouperTemp}{\value{subsection}}\Ssubsubsectionstar{#1}\setcounter{subsection}{\value{GrouperTemp}}}
\renewcommand{\Ssubsubsubsectiongrouperstar}[1]{\setcounter{GrouperTemp}{\value{subsubsection}}\Ssubsubsectionstar{#1}\setcounter{subsubsection}{\value{GrouperTemp}}}
\renewcommand{\Ssubsubsubsubsectiongrouperstar}[1]{\Ssubsubsubsubsectionstar{#1}}

% To increment section numbers:
\renewcommand{\Sincpart}{\stepcounter{part}}
\renewcommand{\Sincsection}{\stepcounter{chapter}}
\renewcommand{\Sincsubsection}{\stepcounter{section}}
\renewcommand{\Sincsubsubsection}{\stepcounter{subsection}}
\renewcommand{\Sincsubsubsubsection}{\stepcounter{subsubsection}}

% style for page reference
\renewcommand{\PageRef}[1]{第~#1~页}
% style for chapter and section reference
\renewcommand{\ChapRef}[2]{第~#1章}
\renewcommand{\ChapRefUC}[2]{第~#1章}
\renewcommand{\SecRef}[2]{~#1节}
% style for chapter titles:
% 1. title page has no header and footer
% 2. both numbered and unnumbered title should appear in headers
% 3. both numbered and unnumbered subsection should have correct book marks
\renewcommand{\Ssection}[2]{\chapter[#1]{#2}\thispagestyle{empty}}
\renewcommand{\Ssectionstar}[1]{\chapter*{#1}\markboth{#1}{#1}\thispagestyle{empty}}
\renewcommand{\Ssectionstarx}[2]{\Ssectionstar{#2}\addcontentsline{toc}{chapter}{#1}\markboth{#1}{#1}\thispagestyle{empty}}
\renewcommand{\Ssubsectionstarx}[2]{\phantomsection\Ssubsectionstar{#2}\addcontentsline{toc}{section}{#1}}
\renewcommand{\Ssubsubsectionstarx}[2]{\phantomsection\Ssubsubsectionstar{#2}\addcontentsline{toc}{subsection}{#1}}

\renewcommand{\proofname}{证明}
\renewcommand{\contentsname}{目录}

% cancel <, >, and | font encoding
\renewcommand{\Stttextmore}{>}
\renewcommand{\Stttextless}{<}
\renewcommand{\Stttextbar}{|}
